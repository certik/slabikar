% csf-enc.tex

% This file can be used with Unicode engines if you want to use
% non-Unicode CS-font encoded font.

% After \input csf-enc there is the command \csfenc ready to use.
% For example:
%
% \pdfmapline{=slabikar slabikar <slabikar.pfb}
% \font\s=slabikar at25pt

% {\s\baselineskip=35pt\csfenc Tady píšu fontem slabikář česky.\par}
% \bye 
%
% The chachacters listed below are set as active when \csfenc macro is called. 
% Use \csfenc in a group in order to be able to return to normal Unicode encoding.


\def\charenc #1 #2 {\catcode`#1=13 \bgroup \lccode`\~=`#1 \lowercase{\egroup \chardef~=#2}}

\def\csfenc{%
   \charenc á 225 % a-acute
   \charenc Á 193 % A-acute
   \charenc ä 228 % a-diaeresis
   \charenc Ä 196 % A-diaeresis
   \charenc č 232 % c-caron
   \charenc Č 200 % C-caron
   \charenc ď 239 % d-caron
   \charenc Ď 207 % D-caron
   \charenc é 233 % e-acute
   \charenc É 201 % E-acute
   \charenc ě 236 % e-caron
   \charenc Ě 204 % E-caron
   \charenc í 237 % i-acute
   \charenc Í 205 % I-acute
   \charenc ĺ 229 % l-acute
   \charenc Ĺ 197 % L-acute
   \charenc ľ 181 % l-caron
   \charenc Ľ 165 % L-caron
   \charenc ň 242 % n-caron
   \charenc Ň 210 % N-caron
   \charenc ó 243 % o-acute
   \charenc Ó 211 % O-acute
   \charenc ô 244 % o-circumflex
   \charenc Ô 212 % O-circumflex
   \charenc ö 246 % o-diaeresis
   \charenc Ö 214 % O-diaeresis
   \charenc ŕ 224 % r-acute
   \charenc Ŕ 192 % R-acute
   \charenc ř 248 % r-caron
   \charenc Ř 216 % R-caron
   \charenc š 185 % s-caron
   \charenc Š 169 % S-caron
   \charenc ť 187 % t-caron
   \charenc Ť 171 % T-caron
   \charenc ú 250 % u-acute
   \charenc Ú 218 % U-acute
   \charenc ů 249 % u-ring
   \charenc Ů 217 % U-ring
   \charenc ü 252 % u-diaeresis
   \charenc Ü 220 % U-diaeresis
   \charenc ý 253 % y-acute
   \charenc Ý 221 % Y-acute
   \charenc ž 190 % z-caron
   \charenc Ž 174 % Z-caron
   \charenc „ 254 % \clqq
   \charenc “ 255 % \crqq
}

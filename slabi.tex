
\pdfmapline{=slabikar slabikar <slabikar.pfb}

\def\titul #1\par{{\bf#1}\par\nobreak\medskip}
\def\autor #1\par{{\it #1}\medskip}
\def\sub #1\par{\bigskip\noindent{\bf#1}\par\nobreak\medskip}

\chardef\\=`\\
\def\,{\thinspace}

\font\mflogo=logo10
\def\mf{{\mflogo META}\-{\mflogo FONT}}

\hsize=30pc \vsize=46pc \emergencystretch=2em \hbadness=2000
\chyph

{\obeyspaces \gdef\activespace{\obeyspaces\let =\ }}
\def\setverb{\def\do##1{\catcode`##1=12}\dospecials}
\def\begtt{\medskip\bgroup \setverb \activespace
  \def\par##1{\endgraf\ifx##1\par\leavevmode\fi ##1} 
  \obeylines \startverb} 
{\catcode`\|=0 \catcode`\\=12
  |gdef|startverb#1\endtt{|tt#1|egroup|medskip}}

\font\slabikar=slabikar at1cm
\def\ukaz#1{\leavevmode\hbox{\noboundary #1\noboundary\/}}


\titul Psané písmo ze slabikáře

\autor Petr Olšák

Můj syn Mirek začal chodit do první třídy a učí se tedy psát.
Rozhodl jsem se, že naučím psát počítač stejným písmem, jaké se zrovna
nyní Mirek učí. Znamenalo to vytvořit font v~\mf{}u s~psaným písmem
podle slabikáře a zavést ligatury tak, aby písmena pěkně 
k~sobě navazovala.

{\slabikar \baselineskip=1cm \lineskiplimit=-1cm  \def\\{\hfil\break}
\leftskip=\parindent  \noindent
Odemyky-zamyky,\\ rozvažte se, jazyky!\\ Povězte mi v~řeči lidí,\\
co kdo slyší, co kdo vidí.\\ A~z~těch slabik, slov a vět\\
vykouzlíme celý svět.
\par\bigskip}

Přitom stačí do dokumentu napsat:

\begtt
\font\pis=slabikar at 1cm
\def\\{\hfil\break}
\baselineskip=1cm  \pis \noindent
Odemyky-zamyky,\\ rozvažte se, jazyky!\\ 
Povězte mi v řeči lidí,\\ co kdo slyší, co kdo vidí.\\ 
A z~těch slabik, slov a vět\\ vykouzlíme celý svět.
\end
\endtt

K~vytvoření tohoto písma jsem s~výhodou využil tzv.~\uv{zobecněných
ligatur} a hraničních znaků. Aniž by se o~tom příliš vědělo, i takové
vymoženosti nám \TeX{} poskytuje. V~tomto článku se pokusím zmíněné
mechanismy podrobně popsat a ukázat jejich použití na příkladu písma
ze slabikáře.

\sub Použité vlastnosti \TeX{}u

Donald Knuth zavedl algoritmy pro zobecněné ligatury a pro hraniční znaky
až od verze \TeX{}u 3.0. Když doplňoval \TeX{}book~[1] pro tuto novou
verzi, pokusil se neměnit stránkování původní verze této knihy a přitom
tam zanést informace o~nových algoritmech. Protože novinek ve verzi~3.0
bylo více, znamenalo to vyvinout nadlidské úsilí, aby ty stránky zůstaly
v~původním stavu. Pokud člověk přesně zná rozdíly mezi verzí 2.x a 3.0, pak
mu při četbě \TeX{}booku neujde, že se autor snažil některé formulace
k~novým algoritmům až nemístně stlačit do malého prostoru. Konečně, na
některé věci nevybylo místo vůbec. V~rejstříku například najdeme heslo
{\tt\\noboundary}, ale k~němu pouze stručnou zmínku v~závorce o~tom, že:
>>Ligatures and kerns may be influenced by invisible ``boundary''
characters at the left and right, unless {\tt\\noboundary} appears<<. Co to
ty ``boundary'' (hraniční) znaky jsou, se už nedovíme vůbec. K~tomu je
potřeba se podívat do zdrojového textu \TeX{}u~[2] nebo do~[4], kde na
straně 306 píšu:

>>\TeX{} může pracovat se dvěma hraničními znaky: pravý a levý. Pravý klade
(jen teoreticky) na konec každého slova (souvislé posloupnosti znaků ze
stejného fontu bez mezer). Levý hraniční znak se klade analogicky na
začátek každého slova. Tyto znaky fakticky nejsou sázeny, ale mohou ovlivnit
sazbu při vyhodnocování ligatur nebo implicitních kernů s~okrajovým
skutečně viditelným znakem ve slově. Viditelný znak $\alpha$ se může na
začátku slova proměnit v~ligaturu nebo před ním může být implicitní kern,
pokud levý hraniční znak má ve svém odkazu do ligačního a kerningového
programu test na znak $\alpha$. Rovněž se znak $\alpha$ na konci slova může
proměnit v~ligaturu nebo za ním může být implicitní kern, pokud tento
znak má odkaz do ligačního a kerningového programu, kde je test na pravý
hraniční znak.<<

Levý hraniční znak je v~metrice fontu charakterizován ukazatelem do
ligačního a kerningového programu, zatímco pravému hraničnímu znaku
musí být přiřazen nějaký kód. Použití tohoto kódu v~ligačním a
kerningovém programu pak znamená, že se odkazujeme na pravý hraniční znak.

V~\mf{}u je pro levý hraniční znak zaveden speciální symbol~\uv{{\tt ||:}},
který se použije jako návěští v~záznamu {\tt ligtable}. 
Není-li toto návěští nikde použito, \TeX{} s~levým hraničním znakem
nepracuje. 
Pro případ pravého hraničního znaku je nutno v~\mf{}u naplnit primitivní
proměnnou {\tt boundarychar} zvoleným kódem. Tento kód pak můžeme
používat v~záznamu {\tt ligtable}. Není-li tato proměnná naplněna
nebo má hodnotu mimo rozsah 0--255, pak \TeX{} s~pravým hraničním
znakem nepracuje.

Kromě hraničních znaků je \TeX{} vybaven schopností pracovat
s~tzv.~\uv{zobecněnými ligaturami}. Informace o~nich
můžeme rovněž hledat v~[2] nebo~[4].
Nechť dvojice znaků $\alpha\beta$ tvoří ligaturu $\gamma$. Pak
\TeX{} nejprve teoreticky vytvoří trojici $\alpha\gamma\beta$ a dále
podle dodatečných údajů v~metrice
rozhodne o~tom, zda $\alpha$ či $\beta$ či oba tyto znaky
ze sazby vyhodí.

V~\mf{}booku~[3] pak najdeme možnosti, jak tyto zobecněné ligatury
zapisovat do záznamu {\tt ligtable} v~\mf{}ovém textu. Všechny případy
jsou zahrnuty do následující tabulky:

$$\vbox{
\halign{\tt$\alpha$: $\beta$ # $\gamma$\hfil&\hfil\qquad$#$
 \hfil&\qquad\hfil$#$\hfil\cr
 \omit \tt ligtable  & \hbox{výsledná sekvence}
   & \unskip\hbox{případné další změny} \cr
   \noalign{\medskip\hrule\medskip}
     |=:|    &  \alpha\gamma\beta &  \alpha\gamma,  \gamma\beta \cr
     |=:|>   &  \alpha\gamma\beta &  \gamma\beta \cr
     |=:|>>  &  \alpha\gamma\beta &  \cr
    \ =:|    &        \gamma\beta &  \gamma\beta \cr
    \ =:|>   &        \gamma\beta &  \cr
     |=:     &  \alpha\gamma      &  \alpha\gamma \cr
     |=:>    &  \alpha\gamma      &  \cr
    \ =:      &  \gamma            & \cr}
}$$

Všimneme si, že \uv{obyčejná} ligatura, jakou jistě čtenáři znají
z~notorického příkladu \uv{fi}, je realizována zápisem \uv{{\tt =:}} podle
posledního řádku tabulky. V~tomto případě se dvojice \uv{f} a \uv{i}
promění v~ligaturu \uv{fi} a původní písmena \uv{f} a \uv{i} jsou ze
sazby odstraněna. Ostatní řádky nabízejí další, rozšířené možnosti.

V~posledním sloupci tabulky jsou uvedeny dvojice, které dále podléhají
testu na zobecněnou ligaturu nebo implicitní kern. Je-li například
v~{\tt ligtable} použit operátor \uv{{\tt |:=|>}}, pak kromě zařazení zobecněné
ligatury $\gamma$ mezi znaky $\alpha$ a $\beta$ se bude dále testovat, zda
dvojice $\gamma\beta$ netvoří nějakou ligaturu nebo implicitní kern.
S~touto vlastností je potřeba pracovat opatrně, protože může dojít
k~rekurzivnímu tvoření zobecněných ligatur až po zaplnění veškeré paměti
\TeX{}u.

\sub Realizace psaného písma

V~uvedeném písmu podle slabikáře jsem využil jednak vlastnosti hraničních
znaků a jednak zobecněných ligatur. Písmena malé abecedy mají metriky i
jednotlivé tahy voleny tak, aby znaky na sebe navazovaly v~šesti sedminách
střední výšky písma a aby v~tomto bodě měly smluvený, pro všechny dvojice
písmen stejný, sklon. Příklad:

{\baselineskip=1cm \lineskiplimit=-1cm  \def\\{\hfil\break}
\def\ukaz#1{\setbox0=\hbox{\noboundary #1\noboundary\/}\leavevmode
  \lower\dp0\vbox{\hrule height.2pt
    \hbox{\vrule width.2pt\box0\vrule width.2pt\kern-.2pt}%
    \hrule height.2pt\kern-.2pt}}
\leftskip=\parindent  \noindent
{\slabikar rozvažte} \raise3pt\hbox{=}
{\slabikar \ukaz{\char3} \ukaz r \ukaz o~\ukaz z~\ukaz
v~\ukaz a \ukaz ž \ukaz t \kern3pt\ukaz e \ukaz{\char1}
\par\bigskip}}

Každé slovo začínající malým písmenem (s~výjimkou písmen a, c, d, g, o,~q)
začíná náběhovou čárkou: {\slabikar\ukaz{\char3}} nebo:
{\slabikar\ukaz{\char2}}. Jednotlivá písmena už tuto čárku nemají.
Například písmeno \uv{e} z~tohoto důvodu vypadá dost nezvykle~-- spíše jako
\uv{c}. Jednotlivá písmena na sebe navazují bez výjimek přesně podle svých
metrik (metriky znaků jsou v~ukázce vyznačeny obdélníčkem). Nakonec každé
slovo ukončíme maličkou dotahovou čárečkou: {\slabikar\ukaz{\char1}}. Bez
ní by totiž tah končil v~šesti sedminách střední výšky písma. Ačkoli by to
možná z~estetického hlediska bylo lepší, prvňák si to nemůže dovolit a musí
dokončovat tahy až na střední dotažnici. Naším úkolem bylo vytvořit písmo
shodné s~psaným písmem ve slabikáři bez kompromisů.

Náběhová i dotahová čárka bude ke každému slovu připojena automaticky
prostřednictvím mechanismu ligatur a hraničních znaků. Jak již bylo
řečeno, pro levý hraniční znak použijeme v~\mf{}u v~záznamu 
{\tt ligtable} návěští \uv{{\tt ||:}}. Například:

\begtt
ligtable  ||: "-":
      ...,
      "r" |=:|> 3,  rcaron |=:|> 3,
      ...;
\endtt
\noindent
znamená, že levý hraniční znak nebo spojovník tvoří s~následujícím
\uv{r} nebo \uv{ř} ligaturu s~kódem 3 (levá náběhová čárka). Přitom jednak
neviditelný hraniční znak nebo spojovník a jednak znak \uv{r} nebo
\uv{ř} v~sazbě zůstávají. Samozřejmě, že ve fontu je v~místě teček
naší ukázky postupně vyjmenovaná celá abeceda malých
písmen. U~výjimečných písmen, která nepotřebují náběhovou čárku,
můžeme například číst:

\begtt
ligtable  ||: "-":
      "a" kern kk#,  aacute kern kk#,
      ...;
\endtt

Vidíme, že tyto znaky budou pouze posunuty tak, aby byla kompenzována
skutečnost, že vyčnívají vlevo ze svých hranic daných metrikou.

Pro pravý hraniční znak je nutno volit v~\mf{}u nějaký kód pomocí proměnné
{\tt boundarychar}. V~našem fontu byl za pravý hraniční znak zvolen kód~1.
Tento kód má také pravá dotahová čárka, ale sama od sebe se
nevykreslí. Musí se použít mechanismus ligatur. Část {\tt ligtable}, která
řeší tento problém, vypadá takto:

\begtt
boundarychar:=1;
ligtable "a": aacute: "A": Aacute:  "b":
   ...:
   "z": zcaron: "Z": Zcaron:
      1 |=:> 1, "-" |=:|>> 1, "!" |=:|>> 1,
      "?" |=:|>> 1,  "," |=:|>> 1,  "." |=:|>> 1;
\endtt

Místo teček najdete ve skutečném {\tt mf} souboru všechna
písmena, která potřebují dotáhnout na střední dotažnici. Kód
{\tt 1 |=:> 1} zajistí, aby se na konci slov objevila dotahová čárka.
Ostatní kódy zajistí totéž, pokud na konec slova navazuje spojovník,
vykřičník, otazník, čárka a tečka.

Pokud dojde k~automatickému dělení slov, \TeX{} po rozdělení znova
přehodnotí situaci, jak sestavit ligatury. Spojovník na konci řádku proto
způsobí doplnění dotahu na střední dotažnici pro předchozí písmeno. Na
začátku dalšího řádku \TeX{} znova vloží levý hraniční znak a váže na něj
ligatury. Proto se tam objeví správně náběhová čárka.

Velká písmena psané abecedy jsou dvojího druhu. Například \uv{K} se nijak
neliší od malého písmene. Věc je dokonce jednodušší, protože písmeno
nevyžaduje levou náběhovou čárku. Jeho dotah končí v~šesti sedminách
střední výšky písma a může na něj okamžitě navázat další malé písmeno.
Druhým typem je například písmeno~\uv{B}, které v~základním tvaru vypadá tak,
jako bychom jej psali samostatně. Pokud se vyskytne těsně za
písmenem~\uv{B} jakékoli malé písmeno, reaguje tato dvojice vloženou
zobecněnou ligaturou, která obsahuje spojovací čárku:
\kern8pt{\slabikar\ukaz{\char4}}. Samozřejmě, původní znaky, které tuto
ligaturu vyvolaly, v~takové situaci ze sazby nelikvidujeme.

Italické korekce byly vloženy jen k~některým velkým písmenům. Metriky těchto
písmen jsou totiž navrženy s~ohledem na to, aby přesně navázalo následující
malé písmeno (prostřednictvím spojovací čárky). Hranice kresby těchto písmen
proto nemusí odpovídat metrikám. To začne vadit, až budeme klást velká písmena
těsně vedle sebe (například ve zkratkách). V~takovém případě doporučuji
vyvolat italickou korekci. Písmo totiž není primárně navrženo pro psaní
pouze velkými písmeny.

Za zmínku stojí ještě možnost automaticky vyrovnávat některé dvojice malých
písmen.

{\baselineskip=1cm \lineskiplimit=-1cm  \def\\{\hfil\break}
\leftskip=\parindent  \noindent
{\slabikar o\noboundary\kern0pt\noboundary meleta } lépe:
{\slabikar\ omeleta.}
\par\bigskip}

Vlevo vidíme nevyrovnané \uv{o} a \uv{m} (stojí opticky daleko od sebe).
V~pravo je již tato dvojice vyrovnána. S~tabulkou kerningových párů si tady
nepomůžeme. Proto jsem pro tyto případy vytvořil alternativní zúžené \uv{o}
(a také b, v~a w). Tyto alternativy mají kratší dotah. Do tabulky ligatur
jsem zavedl pro dvojice \hbox{(b,o,v,w)--(m,n,v,w,y)} zobecněnou ligaturu,
která likviduje jen levý znak a nahrazuje jej zúženou alternativou.

V~abecedě malých písmen je jeden zajímavý znak. Je jím písmeno \uv{s}.
Toto písmeno má ve fontu několik alternativ. 

{\baselineskip=.8cm \lineskiplimit=-1cm 
základní tvar:  {\slabikar\ \ \ukaz s},       např. {\slabikar les}, 

hlubší tvar:    {\slabikar\ \ \ukaz{\char12}} např. {\slabikar jsme},

plný tvar:      {\slabikar\ \ukaz{\char10}} např. {\slabikar se}.
\par\bigskip}

Základní tvar je použitelný za všemi písmeny, která mají
dotah stejného tvaru, jako například písmeno \uv{m}. Kromě toho existuje
\uv{hlubší} varianta písmene \uv{s}, která se připojuje za písmena končící
spodní kličkou: G, g, J, j, Q, q, Y, y.  Nakonec ještě existuje samostatná
plná varianta písmene \uv{s}. Tato varianta se použije
zcela na začátku slova. Experimenty totiž ukázaly, že pokud sestavíme toto
písmeno pomocí běžné náběhové čárky, pak výsledek působí na začátku slova
poněkud přiťáple. Proměnu písmene \uv{s} na začátku slova v~plnou variantu
zařídíme pomocí ligatury s~levým hraničním znakem. Proměnu za písmenem
s~kličkou v~hlubší variantu zařídíme pomocí zobecněné ligatury. Všechny
varianty písmene \uv{s} automaticky připojují pomocí zobecněné ligatury
spojovací čárku na další písmeno. Používá se stejný mechanismus, jako při
vložení spojovací čárky třeba za velkým písmenem~\uv{B}. Přitom podle tvaru
následujícího písmene jsou ve fontu připraveny různě dlouhé spojovací
čárky. Naprosto stejné alternativy jako~\uv{s} má samozřejmě
písmeno~\uv{š}.


\sub Několik postřehů

Písmo pro první třídu navrhli naši předkové tak, aby bylo jednoduché. Je
tedy postaveno pouze na několika základních principech. Mezi tyto principy
patří naprosto stejné napojování pro zcela všechna písmena.
Jako obvykle platí: v~jednoduchosti je síla.

Toto písmo tvoří základ, ze kterého se pak vyvine rukopis každého z~nás.
Právě pro svou jednoduchost je do velmi dobrý základ. Je ovšem pravda, že
někteří jedinci nemají v~podstatě žádný rukopis. S~přibývajícím množstvím
klávesnic bude těchto jedinců pravděpodobně přibývat. A~to je škoda. Všiml
jsem si, že tyto bytosti bez rukopisu se poměrně hojně vyskytují v~řadách
studentů ČVUT. Většinou jsou to dost velcí nešťastníci, protože si nejsou
schopni zaznamenat ani přednášku. Občas se nějaký student na cvičení
z~matematiky brání na tabuli napsat souvislé slovo, protože mylně předpokládá,
že v~matematice se to nesluší a že stačí psát speciální značky. V~takovém
případě jej většinou požádám, aby napsal aspoň zadání příkladu slovy.
Objeví-li se na tabuli hůlkové písmo, pak si pomyslím, co se ten student
vlastně učil v~první třídě. Podle grafických projevů z~písemných prací
studentů se přitom mohu přesvědčit, že takových lidí je více. Bohužel.

Jako vzor při návrhu písma jsem čerpal ze slabikáře~[5]. I~tyto vzory
podléhají vývoji, takže můžeme nalézt v~různých slabikářích nepatrné
odlišnosti. Například písmeno~\uv{z} se může zdát v~této variantě poněkud
nezvyklé. Zřejmě také záleží na kaligrafovi, který dělal pro příslušný
slabikář vzory psaného písma.

Hotové písmo zveřejňuji v~\mf{}ovém zdroji na Internetu
k~volnému použití. Najdete je na
{\tt http://petr.olsak.net/ftp/olsak/slabikar}.
Písmo je implementováno do jediného souboru {\tt slabikar.mf}.
V souboru {\tt ukazka.tex} najdete jednak ukázku písma a také
doporučení, jak nastavit pro písmo některé parametry \TeX{}u.
Toto písmo je potřeba brát spíš jako příklad, co všechno \TeX{} dovede.
Nepředpokládám velké nasazení tohoto písma pro sazbu příštích
slabikářů. Dokonce takovou věc ani nedoporučuji.

Všechny ukázky ve slabikáři a v~písankách, které jsem měl možnost vidět,
jsou psány lidskou rukou a ne strojem. Samozřejmě smekám před kaligrafem,
který ty ukázky vytvořil. Člověk má na první pohled dojem, že to je
\uv{jak když tiskne}. Fušoval jsem také do kaligrafického řemesla, a proto
dobře vím, že pokud písmo neobsahuje žádné ozdobné prvky, musí to napsat
skutečně profesionál. Každá chybička, která by se třeba skryla za ozdobným
prvkem, je totiž vidět.

Důležité ale je, že písmo v~dnešním slabikáři bylo skutečně napsáno jen
{\it jak\/} když tiskne a nikoli tištěno doopravdy. Písmu tak neschází
lidský rozměr, který ten prvňák podvědomě z~toho písma asi cítí. Kdyby se
pro sazbu ukázek použil stroj (třebaže \TeX), písmo by tento rozměr
ztratilo. Takové písmo by bylo chladné, studené, stále stejné, bez výrazu,
tedy vlastně mrtvé. Nepřeji žádným prvňákům, aby se někdy v~budoucnu
s~takovým chladným písmem setkali. Pokud by snad někoho napadlo mým
\mf{}ovým písmem tisknout písanky nebo slabikáře, budu první, kdo bude
proti. Raději se nabídnu jako kaligraf, který požadované ukázky
napíše rukou.

\sub Literatura

{\hangindent=\parindent \hangafter=1 \parindent=0pt \def\par{{\endgraf}}
\def\[#1] {\leavevmode\hbox to\hangindent{[#1]\hss}}

\[1] Donald Knuth. {\it The \TeX{}book}, volume A~of
{\it Computers \& Typesetting}. Addison-Wesley, Reading, MA, USA, 1989.

\[2] Donald Knuth. {\it \TeX: The program}, volume B of
{\it Computers \& Typesetting}. Addison-Wesley, Reading, MA, USA, 1989.

\[3] Donald Knuth. {\it The \mf{}book}, volume C of
{\it Computers \& Typesetting}. Addison-Wesley, Reading, MA, USA, 1991.

\[4] Petr Olšák. {\it \TeX{}book naruby}, Konvoj, 1997.

\[5] Jiří Žáček, Helena Zmatlíková. {\it Slabikář}, Alter, 1996.

\par}
\end





